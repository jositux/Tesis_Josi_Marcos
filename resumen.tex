%-----------------------------------------------------
% Resúmen
%-----------------------------------------------------
\addcontentsline{toc}{section}{Resúmen}
\section*{Resúmen}
En el pasado la fabricación de un producto consistía en delegar el diseño o entregar los planos a un especialista para que realice la manufactura, los contactos se producían de forma presencial y obligaban a realizar viajes para mantener el trabajo bajo control.
El desarrollo de productos en colaboración es un concepto que está siendo cada vez más utilizado en la industria, especialmente en áreas que involucran el diseño y la fabricación asistida por computadora CAD/CAM. Los procesos de colaboración actual poseen características inherentes a la evolución de las telecomunicaciones y las tecnologías web, se asume que los proyectos contemplan la participación de personas dispersas geográficamente.\vskip
Un equipo multidisciplinario se convierte en un valor agregado para los proyectos, incluso con personas sin conocimientos técnicos en programas avanzados de CAD  porque aportan componentes efectivos para ser integrados en los productos finales. Al mismo tiempo, las diferentes disciplinas producen de forma natural una gran diversidad de capacidades técnicas y por consecuencia, problemas de comunicación al intentar compartir información precisa y eficiente sobre los productos. 
Estos problemas están directamente vinculados al contenido compartido que por lo general implica el uso de modelos 3D como nexo para visualizar, intercambiar ideas y revisar los diseños.\vskip
El presente trabajo de tesis consta en el diseño e implementación un prototipo de software web de uso sencillo que permita resolver los problemas señalados, haciendo posible la colaboración entre usuarios con diferentes conocimientos técnicos mediante revisiones de modelos 3D y establecer una comunicación fluída que vaya más allá de la geometría.\vskip
El prototipo llamado “XXXX” fué diseñado basándose en entrevistas usando el enfoque LEAN UX y desarrollado utilizando metodologías ágiles con tecnologías web FLOOS (Javascript, Node.js, OpenJSCAD, Vue.js, etc).
Para validar el funcionamiento del software se plantearon dos escenarios de laboratorio con usuarios reales y se establecieron las conclusiones pertinentes.
El software está documentado y disponible online para la descarga con una licencia open source en https://github.com/jositux/Tesis_Josi_Marcos\\

Palabras Claves: CAD, CPD, SPA, OpenJSCAD, Javascript, Vue.js, Node.js, LEAN UX, Agile




	
% ADDITIONAL DECLARATIONS HERE (IF ANY)

%\vskip5mm
%Signature:
%\vskip20mm
%AUTOR
%José María Guaimas