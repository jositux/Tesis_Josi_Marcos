%-----------------------------------------------------
% Resúmen
%-----------------------------------------------------
\addcontentsline{toc}{section}{Resúmen}
\section*{Resúmen}
En el pasado el proceso de fabricar un producto consistía en delegar el diseño o enviar planos a un especialista para que realice la manufactura, los contactos era presenciales y era necesario viajar periódicamente para mantener el proyecto bajo control. 
Al aumentar la cantidad de participantes, la toma de decisiones se volvía cada vez más compleja, extendiendo los tiempos de diseño.\vskip

En la actualidad el Desarrollo Colaborativo de Productos y el co-diseño son conceptos muy utilizados en las organizaciones, en especial, en las áreas que involucran el ``diseño y Fabricación Asistida por Computadora'' por sus siglas en inglés \textit{Computer-Aided Design/Computer-Aided Manufacturing} (CAD/CAM). En este contexto y gracias al uso de las tecnologías web actuales se asume la colaboración entre personas dispersas geográficamente y de diferentes campos de especialización. De esta forma los equipos multidisciplinarios se convierten en un valor agregado para los proyectos, incluso con la participación de personas sin instrucción en diseño. \vskip
Al mismo tiempo, la diversidad de conocimientos trae como consecuencia algunos problemas en la gestión de los proyectos, entre ellos: la comunicación imprecisa de los participantes, la múltiple interpretación de ideas y la complejidad en el registro de cambios en los diseños. Estos problemas se relacionan directamente con la estrategia elegida para compartir el contenido de un producto que, por lo general, implica el uso de modelos 3D como nexo para la visualización, el intercambio de opiniones y la revisión de los diseños.\vskip


El presente trabajo propone el diseño e implementación de un prototipo de aplicación web que permita resolver los problemas señalados, haciendo posible la colaboración multidisciplinaria  mediante revisiones de modelos 3D y establecer una comunicación fluída que vaya más allá de la geometría. El prototipo llamado \textit{Colaborative CAD Application} (COCADA) será diseñado en base a entrevistas usando el enfoque LEAN UX y desarrollado utilizando metodologías ágiles con tecnologías web \textit{Free Libre Open Source Software} (FLOSS). \vskip
Para validar el funcionamiento se plantearon dos escenarios de prueba y se establecieron las conclusiones pertinentes.
El software se encuentra disponible en https://github.com/jositux/Tesis_Josi_Marcos bajo una licencia de software libre. 

\vskip
\vskip
Palabras Claves: CAD, CPD, SPA, OpenJSCAD, Javascript, Vue.js, Node.js, LEAN UX, Agile




	
% ADDITIONAL DECLARATIONS HERE (IF ANY)

%\vskip5mm
%Signature:
%\vskip20mm
%AUTOR
%José María Guaimas



%-----------------------------------------------------
% Resúmen 2
%-----------------------------------------------------
\addcontentsline{toc}{section}{Resúmen}
\section*{Resúmen 2 - \textcolor{red}{Preferimos este mas consiso}}
En el pasado el proceso de fabricar un producto consistía en delegar el diseño o enviar planos a un especialista para que realice la manufactura, los contactos eran presenciales y mantener el proyecto bajo control obligaba a realizar viajes frecuentes. 

En la actualidad el contexto es muy diferente: El Desarrollo Colaborativo de Productos y el Co-Diseño son conceptos muy utilizados en las organizaciones, en especial, en las áreas que involucran el diseño y fabricación asistido por computadora usando técnicas CAD/CAM. Gracias a la evolución de las tecnologías web se asume la colaboración entre personas dispersas geográficamente, de diferentes campos de especialización e incluso sin formación en diseño. \vskip
Al mismo tiempo, la diversidad de conocimientos trae como consecuencia algunos problemas en la gestión de los proyectos, entre ellos: la comunicación imprecisa entre los participantes, la múltiple interpretación de ideas y la complejidad en el registro de cambios en los diseños. 

El presente trabajo propone el diseño e implementación de un prototipo de aplicación web que permite la colaboración multidisciplinaria mediante revisiones de modelos 3D, estableciendo una comunicación fluída entre los participantes que vaya más allá de la geometría. 

El prototipo llamado \textit{Colaborative CAD Application} (COCADA) utiliza el enfoque LEAN UX combinado con metodologías ágiles de desarrollo de software y el uso de tecnologías web \textit{Free Libre Open Source Software} (FLOSS)\footnote{\url{https://es.wikipedia.org/wiki/Software\_libre\_y\_de\_c\%C3\%B3digo_abierto}}. \vskip
 \vskip
El software COCADA se encuentra disponible para la descarga en \url{https://github.com/jositux/Tesis_Josi_Marcos}  bajo licencia GNU General Public License (GPL)\footnote{\url{https://www.gnu.org/licenses/gpl-3.0.en.html}}. 

\vskip5mm
Palabras Claves: Diseño Colaborativo, CPD, OpenJSCAD, Javascript, Vue.js, Node.js, LEAN UX, Agile




	
% ADDITIONAL DECLARATIONS HERE (IF ANY)

%\vskip5mm
%Signature:
%\vskip20mm
%AUTOR
%José María Guaimas